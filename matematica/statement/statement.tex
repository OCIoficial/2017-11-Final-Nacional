\documentclass{oci}
\usepackage[utf8]{inputenc}
\usepackage{lipsum}

\title{La tarea de matemática}

\begin{document}
\begin{problemDescription}
  Un fatídico día en que la profesora de matemática ya estaba aburrida de
  hacer clases, no encontró mejor idea que dar como tarea escribir en el
  cuaderno de uno a un millón.

  Después de algunas horas escribiendo los alumnos comenzaron a darse cuenta que
  la tarea era titánica.
  ?`Cuántas hojas serían necesarias para escribir todos los números hasta un
  millón?
  ?`Entraría todo en un solo cuaderno?
  O más aún, ?`sería siguiera posible escribir en el cuaderno hasta cien mil?

  Un cuaderno clásico de cuadro grande tiene \textbf{28 cuadros a lo ancho y 35
    a lo alto}.
  Podemos suponer que en cada cuadro se escribe un solo dígito y que dos pares
  de números consecutivos se separar con una coma que no ocupa espacio.
  Por otro lado un número debe ser escrito en una sola línea y si no hay espacio
  en la línea actual se deben dejar los cuadros restantes en blanco y partir en
  la siguiente.

  Dado un entero $N$ tu tarea es encontrar la cantidad de hojas necesarias para
  escribir todos los números de 1 a $N$.
\end{problemDescription}

\begin{inputDescription}
  La entrada corresponde a un único entero $N$ correspondiente al número hasta
  el cuál se desea escribir.
\end{inputDescription}

\begin{outputDescription}
  La salida debe contener una única línea con un entero correspondiente a la
  cantidad de hojas necesarias para escribir de 1 a $N$ siguiendo las reglas del
  enunciado.
\end{outputDescription}

\begin{scoreDescription}
  \score{0} Se probarán varios casos donde $0 < N < 10$.
  \score{0} Se probarán varios casos donde $10 < N \leq 500$.
  \score{0} Se probarán varios casos donde $500 < N \leq 1000000$.
\end{scoreDescription}

\begin{sampleDescription}
\sampleIO{sample-1}
\sampleIO{sample-2}
\end{sampleDescription}

\end{document}
