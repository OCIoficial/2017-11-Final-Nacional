\documentclass{oci}
\usepackage{array}
\usepackage[utf8]{inputenc}
\usepackage{lipsum}
\usepackage{colortbl}

\title{La tarea de matemática}

\begin{document}
\begin{problemDescription}
  Un fatídico día en que la profesora de matemática ya estaba aburrida de hacer
  clases, no encontró mejor idea que dar a sus alumnos la tarea de escribir en
  el cuaderno los números del uno al un millón.

  Después de algunas horas escribiendo, los alumnos notaron
  que la tarea era titánica. ?`Cuántas hojas serían necesarias para escribir
  todos los números hasta un millón? ?`Entraría todo en un solo cuaderno? 
  ?`Sería siquiera posible escribir en el cuaderno hasta cien mil?

  Un cuaderno clásico de cuadro grande tiene \textbf{28 cuadros a lo ancho y 35
    a lo alto}. Podemos suponer que en cada cuadro se escribe un solo dígito y
  que dos números consecutivos se separan con una coma (que ocupa tan
  poco espacio que es despreciable). Por otro lado un número debe ser escrito completo en
  una sola línea. Si no queda espacio en la línea actual se deben dejar los
  cuadros restantes en blanco y continuar en la siguiente línea (o página).
  La siguiente figura muestra cómo quedaría la primera página si se llena con la
  mayor cantidad de números posible.
  
  \begin{center}
  %\newcolumntype{L}[1]{>{\raggedright\let\newline\\\arraybackslash}m{1em}|}
\scalebox{0.77}{
\def\arraystretch{0.9}
\setlength{\tabcolsep}{0.1em}
\begin{tabular}{!{\color{lightgray}\vrule}!{\color{lightgray}\vrule}c!{\color{lightgray}\vrule}c!{\color{lightgray}\vrule}c!{\color{lightgray}\vrule}c!{\color{lightgray}\vrule}c!{\color{lightgray}\vrule}c!{\color{lightgray}\vrule}c!{\color{lightgray}\vrule}c!{\color{lightgray}\vrule}c!{\color{lightgray}\vrule}c!{\color{lightgray}\vrule}c!{\color{lightgray}\vrule}c!{\color{lightgray}\vrule}c!{\color{lightgray}\vrule}c!{\color{lightgray}\vrule}c!{\color{lightgray}\vrule}c!{\color{lightgray}\vrule}c!{\color{lightgray}\vrule}c!{\color{lightgray}\vrule}c!{\color{lightgray}\vrule}c!{\color{lightgray}\vrule}c!{\color{lightgray}\vrule}c!{\color{lightgray}\vrule}c!{\color{lightgray}\vrule}c!{\color{lightgray}\vrule}c!{\color{lightgray}\vrule}c!{\color{lightgray}\vrule}c!{\color{lightgray}\vrule}c!{\color{lightgray}\vrule}!{\color{lightgray}\vrule}}
\arrayrulecolor{lightgray}\hline
\phantom{,}1,&\phantom{,}2,&\phantom{,}3,&\phantom{,}4,&\phantom{,}5,&\phantom{,}6,&\phantom{,}7,&\phantom{,}8,&\phantom{,}9,&\phantom{,}1\phantom{,}&\phantom{,}0,&\phantom{,}1\phantom{,}&\phantom{,}1,&\phantom{,}1\phantom{,}&\phantom{,}2,&\phantom{,}1\phantom{,}&\phantom{,}3,&\phantom{,}1\phantom{,}&\phantom{,}4,&\phantom{,}1\phantom{,}&\phantom{,}5,&\phantom{,}1\phantom{,}&\phantom{,}6,&\phantom{,}1\phantom{,}&\phantom{,}7,&\phantom{,}1\phantom{,}&\phantom{,}8,&\\
\arrayrulecolor{lightgray}\hline
\phantom{,}1\phantom{,}&\phantom{,}9,&\phantom{,}2\phantom{,}&\phantom{,}0,&\phantom{,}2\phantom{,}&\phantom{,}1,&\phantom{,}2\phantom{,}&\phantom{,}2,&\phantom{,}2\phantom{,}&\phantom{,}3,&\phantom{,}2\phantom{,}&\phantom{,}4,&\phantom{,}2\phantom{,}&\phantom{,}5,&\phantom{,}2\phantom{,}&\phantom{,}6,&\phantom{,}2\phantom{,}&\phantom{,}7,&\phantom{,}2\phantom{,}&\phantom{,}8,&\phantom{,}2\phantom{,}&\phantom{,}9,&\phantom{,}3\phantom{,}&\phantom{,}0,&\phantom{,}3\phantom{,}&\phantom{,}1,&\phantom{,}3\phantom{,}&\phantom{,}2,\\
\arrayrulecolor{lightgray}\hline
\phantom{,}3\phantom{,}&\phantom{,}3,&\phantom{,}3\phantom{,}&\phantom{,}4,&\phantom{,}3\phantom{,}&\phantom{,}5,&\phantom{,}3\phantom{,}&\phantom{,}6,&\phantom{,}3\phantom{,}&\phantom{,}7,&\phantom{,}3\phantom{,}&\phantom{,}8,&\phantom{,}3\phantom{,}&\phantom{,}9,&\phantom{,}4\phantom{,}&\phantom{,}0,&\phantom{,}4\phantom{,}&\phantom{,}1,&\phantom{,}4\phantom{,}&\phantom{,}2,&\phantom{,}4\phantom{,}&\phantom{,}3,&\phantom{,}4\phantom{,}&\phantom{,}4,&\phantom{,}4\phantom{,}&\phantom{,}5,&\phantom{,}4\phantom{,}&\phantom{,}6,\\
\arrayrulecolor{lightgray}\hline
\phantom{,}4\phantom{,}&\phantom{,}7,&\phantom{,}4\phantom{,}&\phantom{,}8,&\phantom{,}4\phantom{,}&\phantom{,}9,&\phantom{,}5\phantom{,}&\phantom{,}0,&\phantom{,}5\phantom{,}&\phantom{,}1,&\phantom{,}5\phantom{,}&\phantom{,}2,&\phantom{,}5\phantom{,}&\phantom{,}3,&\phantom{,}5\phantom{,}&\phantom{,}4,&\phantom{,}5\phantom{,}&\phantom{,}5,&\phantom{,}5\phantom{,}&\phantom{,}6,&\phantom{,}5\phantom{,}&\phantom{,}7,&\phantom{,}5\phantom{,}&\phantom{,}8,&\phantom{,}5\phantom{,}&\phantom{,}9,&\phantom{,}6\phantom{,}&\phantom{,}0,\\
\arrayrulecolor{lightgray}\hline
\phantom{,}6\phantom{,}&\phantom{,}1,&\phantom{,}6\phantom{,}&\phantom{,}2,&\phantom{,}6\phantom{,}&\phantom{,}3,&\phantom{,}6\phantom{,}&\phantom{,}4,&\phantom{,}6\phantom{,}&\phantom{,}5,&\phantom{,}6\phantom{,}&\phantom{,}6,&\phantom{,}6\phantom{,}&\phantom{,}7,&\phantom{,}6\phantom{,}&\phantom{,}8,&\phantom{,}6\phantom{,}&\phantom{,}9,&\phantom{,}7\phantom{,}&\phantom{,}0,&\phantom{,}7\phantom{,}&\phantom{,}1,&\phantom{,}7\phantom{,}&\phantom{,}2,&\phantom{,}7\phantom{,}&\phantom{,}3,&\phantom{,}7\phantom{,}&\phantom{,}4,\\
\arrayrulecolor{lightgray}\hline
\phantom{,}7\phantom{,}&\phantom{,}5,&\phantom{,}7\phantom{,}&\phantom{,}6,&\phantom{,}7\phantom{,}&\phantom{,}7,&\phantom{,}7\phantom{,}&\phantom{,}8,&\phantom{,}7\phantom{,}&\phantom{,}9,&\phantom{,}8\phantom{,}&\phantom{,}0,&\phantom{,}8\phantom{,}&\phantom{,}1,&\phantom{,}8\phantom{,}&\phantom{,}2,&\phantom{,}8\phantom{,}&\phantom{,}3,&\phantom{,}8\phantom{,}&\phantom{,}4,&\phantom{,}8\phantom{,}&\phantom{,}5,&\phantom{,}8\phantom{,}&\phantom{,}6,&\phantom{,}8\phantom{,}&\phantom{,}7,&\phantom{,}8\phantom{,}&\phantom{,}8,\\
\arrayrulecolor{lightgray}\hline
\phantom{,}8\phantom{,}&\phantom{,}9,&\phantom{,}9\phantom{,}&\phantom{,}0,&\phantom{,}9\phantom{,}&\phantom{,}1,&\phantom{,}9\phantom{,}&\phantom{,}2,&\phantom{,}9\phantom{,}&\phantom{,}3,&\phantom{,}9\phantom{,}&\phantom{,}4,&\phantom{,}9\phantom{,}&\phantom{,}5,&\phantom{,}9\phantom{,}&\phantom{,}6,&\phantom{,}9\phantom{,}&\phantom{,}7,&\phantom{,}9\phantom{,}&\phantom{,}8,&\phantom{,}9\phantom{,}&\phantom{,}9,&\phantom{,}1\phantom{,}&\phantom{,}0\phantom{,}&\phantom{,}0,&\phantom{,}1\phantom{,}&\phantom{,}0\phantom{,}&\phantom{,}1,\\
\arrayrulecolor{lightgray}\hline
\phantom{,}1\phantom{,}&\phantom{,}0\phantom{,}&\phantom{,}2,&\phantom{,}1\phantom{,}&\phantom{,}0\phantom{,}&\phantom{,}3,&\phantom{,}1\phantom{,}&\phantom{,}0\phantom{,}&\phantom{,}4,&\phantom{,}1\phantom{,}&\phantom{,}0\phantom{,}&\phantom{,}5,&\phantom{,}1\phantom{,}&\phantom{,}0\phantom{,}&\phantom{,}6,&\phantom{,}1\phantom{,}&\phantom{,}0\phantom{,}&\phantom{,}7,&\phantom{,}1\phantom{,}&\phantom{,}0\phantom{,}&\phantom{,}8,&\phantom{,}1\phantom{,}&\phantom{,}0\phantom{,}&\phantom{,}9,&\phantom{,}1\phantom{,}&\phantom{,}1\phantom{,}&\phantom{,}0,&\\
\arrayrulecolor{lightgray}\hline
\phantom{,}1\phantom{,}&\phantom{,}1\phantom{,}&\phantom{,}1,&\phantom{,}1\phantom{,}&\phantom{,}1\phantom{,}&\phantom{,}2,&\phantom{,}1\phantom{,}&\phantom{,}1\phantom{,}&\phantom{,}3,&\phantom{,}1\phantom{,}&\phantom{,}1\phantom{,}&\phantom{,}4,&\phantom{,}1\phantom{,}&\phantom{,}1\phantom{,}&\phantom{,}5,&\phantom{,}1\phantom{,}&\phantom{,}1\phantom{,}&\phantom{,}6,&\phantom{,}1\phantom{,}&\phantom{,}1\phantom{,}&\phantom{,}7,&\phantom{,}1\phantom{,}&\phantom{,}1\phantom{,}&\phantom{,}8,&\phantom{,}1\phantom{,}&\phantom{,}1\phantom{,}&\phantom{,}9,&\\
\arrayrulecolor{lightgray}\hline
\phantom{,}1\phantom{,}&\phantom{,}2\phantom{,}&\phantom{,}0,&\phantom{,}1\phantom{,}&\phantom{,}2\phantom{,}&\phantom{,}1,&\phantom{,}1\phantom{,}&\phantom{,}2\phantom{,}&\phantom{,}2,&\phantom{,}1\phantom{,}&\phantom{,}2\phantom{,}&\phantom{,}3,&\phantom{,}1\phantom{,}&\phantom{,}2\phantom{,}&\phantom{,}4,&\phantom{,}1\phantom{,}&\phantom{,}2\phantom{,}&\phantom{,}5,&\phantom{,}1\phantom{,}&\phantom{,}2\phantom{,}&\phantom{,}6,&\phantom{,}1\phantom{,}&\phantom{,}2\phantom{,}&\phantom{,}7,&\phantom{,}1\phantom{,}&\phantom{,}2\phantom{,}&\phantom{,}8,&\\
\arrayrulecolor{lightgray}\hline
\phantom{,}1\phantom{,}&\phantom{,}2\phantom{,}&\phantom{,}9,&\phantom{,}1\phantom{,}&\phantom{,}3\phantom{,}&\phantom{,}0,&\phantom{,}1\phantom{,}&\phantom{,}3\phantom{,}&\phantom{,}1,&\phantom{,}1\phantom{,}&\phantom{,}3\phantom{,}&\phantom{,}2,&\phantom{,}1\phantom{,}&\phantom{,}3\phantom{,}&\phantom{,}3,&\phantom{,}1\phantom{,}&\phantom{,}3\phantom{,}&\phantom{,}4,&\phantom{,}1\phantom{,}&\phantom{,}3\phantom{,}&\phantom{,}5,&\phantom{,}1\phantom{,}&\phantom{,}3\phantom{,}&\phantom{,}6,&\phantom{,}1\phantom{,}&\phantom{,}3\phantom{,}&\phantom{,}7,&\\
\arrayrulecolor{lightgray}\hline
\phantom{,}1\phantom{,}&\phantom{,}3\phantom{,}&\phantom{,}8,&\phantom{,}1\phantom{,}&\phantom{,}3\phantom{,}&\phantom{,}9,&\phantom{,}1\phantom{,}&\phantom{,}4\phantom{,}&\phantom{,}0,&\phantom{,}1\phantom{,}&\phantom{,}4\phantom{,}&\phantom{,}1,&\phantom{,}1\phantom{,}&\phantom{,}4\phantom{,}&\phantom{,}2,&\phantom{,}1\phantom{,}&\phantom{,}4\phantom{,}&\phantom{,}3,&\phantom{,}1\phantom{,}&\phantom{,}4\phantom{,}&\phantom{,}4,&\phantom{,}1\phantom{,}&\phantom{,}4\phantom{,}&\phantom{,}5,&\phantom{,}1\phantom{,}&\phantom{,}4\phantom{,}&\phantom{,}6,&\\
\arrayrulecolor{lightgray}\hline
\phantom{,}1\phantom{,}&\phantom{,}4\phantom{,}&\phantom{,}7,&\phantom{,}1\phantom{,}&\phantom{,}4\phantom{,}&\phantom{,}8,&\phantom{,}1\phantom{,}&\phantom{,}4\phantom{,}&\phantom{,}9,&\phantom{,}1\phantom{,}&\phantom{,}5\phantom{,}&\phantom{,}0,&\phantom{,}1\phantom{,}&\phantom{,}5\phantom{,}&\phantom{,}1,&\phantom{,}1\phantom{,}&\phantom{,}5\phantom{,}&\phantom{,}2,&\phantom{,}1\phantom{,}&\phantom{,}5\phantom{,}&\phantom{,}3,&\phantom{,}1\phantom{,}&\phantom{,}5\phantom{,}&\phantom{,}4,&\phantom{,}1\phantom{,}&\phantom{,}5\phantom{,}&\phantom{,}5,&\\
\arrayrulecolor{lightgray}\hline
\phantom{,}1\phantom{,}&\phantom{,}5\phantom{,}&\phantom{,}6,&\phantom{,}1\phantom{,}&\phantom{,}5\phantom{,}&\phantom{,}7,&\phantom{,}1\phantom{,}&\phantom{,}5\phantom{,}&\phantom{,}8,&\phantom{,}1\phantom{,}&\phantom{,}5\phantom{,}&\phantom{,}9,&\phantom{,}1\phantom{,}&\phantom{,}6\phantom{,}&\phantom{,}0,&\phantom{,}1\phantom{,}&\phantom{,}6\phantom{,}&\phantom{,}1,&\phantom{,}1\phantom{,}&\phantom{,}6\phantom{,}&\phantom{,}2,&\phantom{,}1\phantom{,}&\phantom{,}6\phantom{,}&\phantom{,}3,&\phantom{,}1\phantom{,}&\phantom{,}6\phantom{,}&\phantom{,}4,&\\
\arrayrulecolor{lightgray}\hline
\phantom{,}1\phantom{,}&\phantom{,}6\phantom{,}&\phantom{,}5,&\phantom{,}1\phantom{,}&\phantom{,}6\phantom{,}&\phantom{,}6,&\phantom{,}1\phantom{,}&\phantom{,}6\phantom{,}&\phantom{,}7,&\phantom{,}1\phantom{,}&\phantom{,}6\phantom{,}&\phantom{,}8,&\phantom{,}1\phantom{,}&\phantom{,}6\phantom{,}&\phantom{,}9,&\phantom{,}1\phantom{,}&\phantom{,}7\phantom{,}&\phantom{,}0,&\phantom{,}1\phantom{,}&\phantom{,}7\phantom{,}&\phantom{,}1,&\phantom{,}1\phantom{,}&\phantom{,}7\phantom{,}&\phantom{,}2,&\phantom{,}1\phantom{,}&\phantom{,}7\phantom{,}&\phantom{,}3,&\\
\arrayrulecolor{lightgray}\hline
\phantom{,}1\phantom{,}&\phantom{,}7\phantom{,}&\phantom{,}4,&\phantom{,}1\phantom{,}&\phantom{,}7\phantom{,}&\phantom{,}5,&\phantom{,}1\phantom{,}&\phantom{,}7\phantom{,}&\phantom{,}6,&\phantom{,}1\phantom{,}&\phantom{,}7\phantom{,}&\phantom{,}7,&\phantom{,}1\phantom{,}&\phantom{,}7\phantom{,}&\phantom{,}8,&\phantom{,}1\phantom{,}&\phantom{,}7\phantom{,}&\phantom{,}9,&\phantom{,}1\phantom{,}&\phantom{,}8\phantom{,}&\phantom{,}0,&\phantom{,}1\phantom{,}&\phantom{,}8\phantom{,}&\phantom{,}1,&\phantom{,}1\phantom{,}&\phantom{,}8\phantom{,}&\phantom{,}2,&\\
\arrayrulecolor{lightgray}\hline
\phantom{,}1\phantom{,}&\phantom{,}8\phantom{,}&\phantom{,}3,&\phantom{,}1\phantom{,}&\phantom{,}8\phantom{,}&\phantom{,}4,&\phantom{,}1\phantom{,}&\phantom{,}8\phantom{,}&\phantom{,}5,&\phantom{,}1\phantom{,}&\phantom{,}8\phantom{,}&\phantom{,}6,&\phantom{,}1\phantom{,}&\phantom{,}8\phantom{,}&\phantom{,}7,&\phantom{,}1\phantom{,}&\phantom{,}8\phantom{,}&\phantom{,}8,&\phantom{,}1\phantom{,}&\phantom{,}8\phantom{,}&\phantom{,}9,&\phantom{,}1\phantom{,}&\phantom{,}9\phantom{,}&\phantom{,}0,&\phantom{,}1\phantom{,}&\phantom{,}9\phantom{,}&\phantom{,}1,&\\
\arrayrulecolor{lightgray}\hline
\phantom{,}1\phantom{,}&\phantom{,}9\phantom{,}&\phantom{,}2,&\phantom{,}1\phantom{,}&\phantom{,}9\phantom{,}&\phantom{,}3,&\phantom{,}1\phantom{,}&\phantom{,}9\phantom{,}&\phantom{,}4,&\phantom{,}1\phantom{,}&\phantom{,}9\phantom{,}&\phantom{,}5,&\phantom{,}1\phantom{,}&\phantom{,}9\phantom{,}&\phantom{,}6,&\phantom{,}1\phantom{,}&\phantom{,}9\phantom{,}&\phantom{,}7,&\phantom{,}1\phantom{,}&\phantom{,}9\phantom{,}&\phantom{,}8,&\phantom{,}1\phantom{,}&\phantom{,}9\phantom{,}&\phantom{,}9,&\phantom{,}2\phantom{,}&\phantom{,}0\phantom{,}&\phantom{,}0,&\\
\arrayrulecolor{lightgray}\hline
\phantom{,}2\phantom{,}&\phantom{,}0\phantom{,}&\phantom{,}1,&\phantom{,}2\phantom{,}&\phantom{,}0\phantom{,}&\phantom{,}2,&\phantom{,}2\phantom{,}&\phantom{,}0\phantom{,}&\phantom{,}3,&\phantom{,}2\phantom{,}&\phantom{,}0\phantom{,}&\phantom{,}4,&\phantom{,}2\phantom{,}&\phantom{,}0\phantom{,}&\phantom{,}5,&\phantom{,}2\phantom{,}&\phantom{,}0\phantom{,}&\phantom{,}6,&\phantom{,}2\phantom{,}&\phantom{,}0\phantom{,}&\phantom{,}7,&\phantom{,}2\phantom{,}&\phantom{,}0\phantom{,}&\phantom{,}8,&\phantom{,}2\phantom{,}&\phantom{,}0\phantom{,}&\phantom{,}9,&\\
\arrayrulecolor{lightgray}\hline
\phantom{,}2\phantom{,}&\phantom{,}1\phantom{,}&\phantom{,}0,&\phantom{,}2\phantom{,}&\phantom{,}1\phantom{,}&\phantom{,}1,&\phantom{,}2\phantom{,}&\phantom{,}1\phantom{,}&\phantom{,}2,&\phantom{,}2\phantom{,}&\phantom{,}1\phantom{,}&\phantom{,}3,&\phantom{,}2\phantom{,}&\phantom{,}1\phantom{,}&\phantom{,}4,&\phantom{,}2\phantom{,}&\phantom{,}1\phantom{,}&\phantom{,}5,&\phantom{,}2\phantom{,}&\phantom{,}1\phantom{,}&\phantom{,}6,&\phantom{,}2\phantom{,}&\phantom{,}1\phantom{,}&\phantom{,}7,&\phantom{,}2\phantom{,}&\phantom{,}1\phantom{,}&\phantom{,}8,&\\
\arrayrulecolor{lightgray}\hline
\phantom{,}2\phantom{,}&\phantom{,}1\phantom{,}&\phantom{,}9,&\phantom{,}2\phantom{,}&\phantom{,}2\phantom{,}&\phantom{,}0,&\phantom{,}2\phantom{,}&\phantom{,}2\phantom{,}&\phantom{,}1,&\phantom{,}2\phantom{,}&\phantom{,}2\phantom{,}&\phantom{,}2,&\phantom{,}2\phantom{,}&\phantom{,}2\phantom{,}&\phantom{,}3,&\phantom{,}2\phantom{,}&\phantom{,}2\phantom{,}&\phantom{,}4,&\phantom{,}2\phantom{,}&\phantom{,}2\phantom{,}&\phantom{,}5,&\phantom{,}2\phantom{,}&\phantom{,}2\phantom{,}&\phantom{,}6,&\phantom{,}2\phantom{,}&\phantom{,}2\phantom{,}&\phantom{,}7,&\\
\arrayrulecolor{lightgray}\hline
\phantom{,}2\phantom{,}&\phantom{,}2\phantom{,}&\phantom{,}8,&\phantom{,}2\phantom{,}&\phantom{,}2\phantom{,}&\phantom{,}9,&\phantom{,}2\phantom{,}&\phantom{,}3\phantom{,}&\phantom{,}0,&\phantom{,}2\phantom{,}&\phantom{,}3\phantom{,}&\phantom{,}1,&\phantom{,}2\phantom{,}&\phantom{,}3\phantom{,}&\phantom{,}2,&\phantom{,}2\phantom{,}&\phantom{,}3\phantom{,}&\phantom{,}3,&\phantom{,}2\phantom{,}&\phantom{,}3\phantom{,}&\phantom{,}4,&\phantom{,}2\phantom{,}&\phantom{,}3\phantom{,}&\phantom{,}5,&\phantom{,}2\phantom{,}&\phantom{,}3\phantom{,}&\phantom{,}6,&\\
\arrayrulecolor{lightgray}\hline
\phantom{,}2\phantom{,}&\phantom{,}3\phantom{,}&\phantom{,}7,&\phantom{,}2\phantom{,}&\phantom{,}3\phantom{,}&\phantom{,}8,&\phantom{,}2\phantom{,}&\phantom{,}3\phantom{,}&\phantom{,}9,&\phantom{,}2\phantom{,}&\phantom{,}4\phantom{,}&\phantom{,}0,&\phantom{,}2\phantom{,}&\phantom{,}4\phantom{,}&\phantom{,}1,&\phantom{,}2\phantom{,}&\phantom{,}4\phantom{,}&\phantom{,}2,&\phantom{,}2\phantom{,}&\phantom{,}4\phantom{,}&\phantom{,}3,&\phantom{,}2\phantom{,}&\phantom{,}4\phantom{,}&\phantom{,}4,&\phantom{,}2\phantom{,}&\phantom{,}4\phantom{,}&\phantom{,}5,&\\
\arrayrulecolor{lightgray}\hline
\phantom{,}2\phantom{,}&\phantom{,}4\phantom{,}&\phantom{,}6,&\phantom{,}2\phantom{,}&\phantom{,}4\phantom{,}&\phantom{,}7,&\phantom{,}2\phantom{,}&\phantom{,}4\phantom{,}&\phantom{,}8,&\phantom{,}2\phantom{,}&\phantom{,}4\phantom{,}&\phantom{,}9,&\phantom{,}2\phantom{,}&\phantom{,}5\phantom{,}&\phantom{,}0,&\phantom{,}2\phantom{,}&\phantom{,}5\phantom{,}&\phantom{,}1,&\phantom{,}2\phantom{,}&\phantom{,}5\phantom{,}&\phantom{,}2,&\phantom{,}2\phantom{,}&\phantom{,}5\phantom{,}&\phantom{,}3,&\phantom{,}2\phantom{,}&\phantom{,}5\phantom{,}&\phantom{,}4,&\\
\arrayrulecolor{lightgray}\hline
\phantom{,}2\phantom{,}&\phantom{,}5\phantom{,}&\phantom{,}5,&\phantom{,}2\phantom{,}&\phantom{,}5\phantom{,}&\phantom{,}6,&\phantom{,}2\phantom{,}&\phantom{,}5\phantom{,}&\phantom{,}7,&\phantom{,}2\phantom{,}&\phantom{,}5\phantom{,}&\phantom{,}8,&\phantom{,}2\phantom{,}&\phantom{,}5\phantom{,}&\phantom{,}9,&\phantom{,}2\phantom{,}&\phantom{,}6\phantom{,}&\phantom{,}0,&\phantom{,}2\phantom{,}&\phantom{,}6\phantom{,}&\phantom{,}1,&\phantom{,}2\phantom{,}&\phantom{,}6\phantom{,}&\phantom{,}2,&\phantom{,}2\phantom{,}&\phantom{,}6\phantom{,}&\phantom{,}3,&\\
\arrayrulecolor{lightgray}\hline
\phantom{,}2\phantom{,}&\phantom{,}6\phantom{,}&\phantom{,}4,&\phantom{,}2\phantom{,}&\phantom{,}6\phantom{,}&\phantom{,}5,&\phantom{,}2\phantom{,}&\phantom{,}6\phantom{,}&\phantom{,}6,&\phantom{,}2\phantom{,}&\phantom{,}6\phantom{,}&\phantom{,}7,&\phantom{,}2\phantom{,}&\phantom{,}6\phantom{,}&\phantom{,}8,&\phantom{,}2\phantom{,}&\phantom{,}6\phantom{,}&\phantom{,}9,&\phantom{,}2\phantom{,}&\phantom{,}7\phantom{,}&\phantom{,}0,&\phantom{,}2\phantom{,}&\phantom{,}7\phantom{,}&\phantom{,}1,&\phantom{,}2\phantom{,}&\phantom{,}7\phantom{,}&\phantom{,}2,&\\
\arrayrulecolor{lightgray}\hline
\phantom{,}2\phantom{,}&\phantom{,}7\phantom{,}&\phantom{,}3,&\phantom{,}2\phantom{,}&\phantom{,}7\phantom{,}&\phantom{,}4,&\phantom{,}2\phantom{,}&\phantom{,}7\phantom{,}&\phantom{,}5,&\phantom{,}2\phantom{,}&\phantom{,}7\phantom{,}&\phantom{,}6,&\phantom{,}2\phantom{,}&\phantom{,}7\phantom{,}&\phantom{,}7,&\phantom{,}2\phantom{,}&\phantom{,}7\phantom{,}&\phantom{,}8,&\phantom{,}2\phantom{,}&\phantom{,}7\phantom{,}&\phantom{,}9,&\phantom{,}2\phantom{,}&\phantom{,}8\phantom{,}&\phantom{,}0,&\phantom{,}2\phantom{,}&\phantom{,}8\phantom{,}&\phantom{,}1,&\\
\arrayrulecolor{lightgray}\hline
\phantom{,}2\phantom{,}&\phantom{,}8\phantom{,}&\phantom{,}2,&\phantom{,}2\phantom{,}&\phantom{,}8\phantom{,}&\phantom{,}3,&\phantom{,}2\phantom{,}&\phantom{,}8\phantom{,}&\phantom{,}4,&\phantom{,}2\phantom{,}&\phantom{,}8\phantom{,}&\phantom{,}5,&\phantom{,}2\phantom{,}&\phantom{,}8\phantom{,}&\phantom{,}6,&\phantom{,}2\phantom{,}&\phantom{,}8\phantom{,}&\phantom{,}7,&\phantom{,}2\phantom{,}&\phantom{,}8\phantom{,}&\phantom{,}8,&\phantom{,}2\phantom{,}&\phantom{,}8\phantom{,}&\phantom{,}9,&\phantom{,}2\phantom{,}&\phantom{,}9\phantom{,}&\phantom{,}0,&\\
\arrayrulecolor{lightgray}\hline
\phantom{,}2\phantom{,}&\phantom{,}9\phantom{,}&\phantom{,}1,&\phantom{,}2\phantom{,}&\phantom{,}9\phantom{,}&\phantom{,}2,&\phantom{,}2\phantom{,}&\phantom{,}9\phantom{,}&\phantom{,}3,&\phantom{,}2\phantom{,}&\phantom{,}9\phantom{,}&\phantom{,}4,&\phantom{,}2\phantom{,}&\phantom{,}9\phantom{,}&\phantom{,}5,&\phantom{,}2\phantom{,}&\phantom{,}9\phantom{,}&\phantom{,}6,&\phantom{,}2\phantom{,}&\phantom{,}9\phantom{,}&\phantom{,}7,&\phantom{,}2\phantom{,}&\phantom{,}9\phantom{,}&\phantom{,}8,&\phantom{,}2\phantom{,}&\phantom{,}9\phantom{,}&\phantom{,}9,&\\
\arrayrulecolor{lightgray}\hline
\phantom{,}3\phantom{,}&\phantom{,}0\phantom{,}&\phantom{,}0,&\phantom{,}3\phantom{,}&\phantom{,}0\phantom{,}&\phantom{,}1,&\phantom{,}3\phantom{,}&\phantom{,}0\phantom{,}&\phantom{,}2,&\phantom{,}3\phantom{,}&\phantom{,}0\phantom{,}&\phantom{,}3,&\phantom{,}3\phantom{,}&\phantom{,}0\phantom{,}&\phantom{,}4,&\phantom{,}3\phantom{,}&\phantom{,}0\phantom{,}&\phantom{,}5,&\phantom{,}3\phantom{,}&\phantom{,}0\phantom{,}&\phantom{,}6,&\phantom{,}3\phantom{,}&\phantom{,}0\phantom{,}&\phantom{,}7,&\phantom{,}3\phantom{,}&\phantom{,}0\phantom{,}&\phantom{,}8,&\\
\arrayrulecolor{lightgray}\hline
\phantom{,}3\phantom{,}&\phantom{,}0\phantom{,}&\phantom{,}9,&\phantom{,}3\phantom{,}&\phantom{,}1\phantom{,}&\phantom{,}0,&\phantom{,}3\phantom{,}&\phantom{,}1\phantom{,}&\phantom{,}1,&\phantom{,}3\phantom{,}&\phantom{,}1\phantom{,}&\phantom{,}2,&\phantom{,}3\phantom{,}&\phantom{,}1\phantom{,}&\phantom{,}3,&\phantom{,}3\phantom{,}&\phantom{,}1\phantom{,}&\phantom{,}4,&\phantom{,}3\phantom{,}&\phantom{,}1\phantom{,}&\phantom{,}5,&\phantom{,}3\phantom{,}&\phantom{,}1\phantom{,}&\phantom{,}6,&\phantom{,}3\phantom{,}&\phantom{,}1\phantom{,}&\phantom{,}7,&\\
\arrayrulecolor{lightgray}\hline
\phantom{,}3\phantom{,}&\phantom{,}1\phantom{,}&\phantom{,}8,&\phantom{,}3\phantom{,}&\phantom{,}1\phantom{,}&\phantom{,}9,&\phantom{,}3\phantom{,}&\phantom{,}2\phantom{,}&\phantom{,}0,&\phantom{,}3\phantom{,}&\phantom{,}2\phantom{,}&\phantom{,}1,&\phantom{,}3\phantom{,}&\phantom{,}2\phantom{,}&\phantom{,}2,&\phantom{,}3\phantom{,}&\phantom{,}2\phantom{,}&\phantom{,}3,&\phantom{,}3\phantom{,}&\phantom{,}2\phantom{,}&\phantom{,}4,&\phantom{,}3\phantom{,}&\phantom{,}2\phantom{,}&\phantom{,}5,&\phantom{,}3\phantom{,}&\phantom{,}2\phantom{,}&\phantom{,}6,&\\
\arrayrulecolor{lightgray}\hline
\phantom{,}3\phantom{,}&\phantom{,}2\phantom{,}&\phantom{,}7,&\phantom{,}3\phantom{,}&\phantom{,}2\phantom{,}&\phantom{,}8,&\phantom{,}3\phantom{,}&\phantom{,}2\phantom{,}&\phantom{,}9,&\phantom{,}3\phantom{,}&\phantom{,}3\phantom{,}&\phantom{,}0,&\phantom{,}3\phantom{,}&\phantom{,}3\phantom{,}&\phantom{,}1,&\phantom{,}3\phantom{,}&\phantom{,}3\phantom{,}&\phantom{,}2,&\phantom{,}3\phantom{,}&\phantom{,}3\phantom{,}&\phantom{,}3,&\phantom{,}3\phantom{,}&\phantom{,}3\phantom{,}&\phantom{,}4,&\phantom{,}3\phantom{,}&\phantom{,}3\phantom{,}&\phantom{,}5,&\\
\arrayrulecolor{lightgray}\hline
\phantom{,}3\phantom{,}&\phantom{,}3\phantom{,}&\phantom{,}6,&\phantom{,}3\phantom{,}&\phantom{,}3\phantom{,}&\phantom{,}7,&\phantom{,}3\phantom{,}&\phantom{,}3\phantom{,}&\phantom{,}8,&\phantom{,}3\phantom{,}&\phantom{,}3\phantom{,}&\phantom{,}9,&\phantom{,}3\phantom{,}&\phantom{,}4\phantom{,}&\phantom{,}0,&\phantom{,}3\phantom{,}&\phantom{,}4\phantom{,}&\phantom{,}1,&\phantom{,}3\phantom{,}&\phantom{,}4\phantom{,}&\phantom{,}2,&\phantom{,}3\phantom{,}&\phantom{,}4\phantom{,}&\phantom{,}3,&\phantom{,}3\phantom{,}&\phantom{,}4\phantom{,}&\phantom{,}4,&\\
\arrayrulecolor{lightgray}\hline
\phantom{,}3\phantom{,}&\phantom{,}4\phantom{,}&\phantom{,}5,&\phantom{,}3\phantom{,}&\phantom{,}4\phantom{,}&\phantom{,}6,&\phantom{,}3\phantom{,}&\phantom{,}4\phantom{,}&\phantom{,}7,&\phantom{,}3\phantom{,}&\phantom{,}4\phantom{,}&\phantom{,}8,&\phantom{,}3\phantom{,}&\phantom{,}4\phantom{,}&\phantom{,}9,&\phantom{,}3\phantom{,}&\phantom{,}5\phantom{,}&\phantom{,}0,&\phantom{,}3\phantom{,}&\phantom{,}5\phantom{,}&\phantom{,}1,&\phantom{,}3\phantom{,}&\phantom{,}5\phantom{,}&\phantom{,}2,&\phantom{,}3\phantom{,}&\phantom{,}5\phantom{,}&\phantom{,}3,&\\
\arrayrulecolor{lightgray}\hline
\end{tabular}
}
  \end{center}
  
  Dado un entero $N$, tu tarea es encontrar la cantidad de páginas necesarias para
  escribir todos los números desde el 1 hasta el $N$.
\end{problemDescription}

\begin{inputDescription}
  La entrada corresponde a un único entero $N$, tal que $N$ es mayor o igual a $1$ y menor o igual a $1.000.000$. 
\end{inputDescription}

\begin{outputDescription}
  La salida debe contener una única línea con un entero correspondiente a la
  cantidad de páginas necesarias para escribir todos los números del 1 al $N$ siguiendo las reglas
  descritas en el enunciado.
\end{outputDescription}

\section*{Subtareas y puntaje}
En este problema no hay subtareas. Se dará 4 puntos por cada caso de prueba
correcto.

\begin{sampleDescription}
  \sampleIO{sample-1} \sampleIO{sample-2} \sampleIO{sample-3}
\end{sampleDescription}

\end{document}
