\documentclass{oci}
\usepackage[utf8]{inputenc}
\usepackage{lipsum}

\title{Transbordo en el aeropuerto}

\begin{document}
\begin{problemDescription}
  A Lucas le encanta viajar.
  Lo hace varias veces al año, escogiendo siempre un nuevo destino para conocer.
  Para ahorrar dinero, y así poder viajar más, Lucas siempre toma los vuelos más
  baratos.
  Esto suele ser un problema pues a menudo tiene que realizar escalas y cambiar
  de avión en muy poco tiempo.

  Los aeropuertos están siempre divididos en \emph{terminales}.
  Cuando Lucas se baja del avión lo hace en un terminal de inicio $S$.
  Luego tiene que recorrer el aeropuerto hasta llegar a un terminal de destino
  $T$ donde debe tomar el siguiente avión.
  Los terminales dentro de un aeropuerto están conectados por caminos que pueden
  ser recorridos caminando en ambos sentidos.
  Después de tantos viajes Lucas ya conoce prácticamente todos los aeropuertos
  y tiene un mapa de cada uno de estos con los caminos que hay entre terminales
  junto con el tiempo que tarda en recorrerlos.

  Muchas veces es imposible recorrer los aeropuertos solo caminando así que
  estos ponen a disposición trenes entre algunos terminales.
  Lucas también ha recolectado la información de estos trenes y tiene una tabla
  de viajes con el momento exacto en que los trenes partirán de cada terminal y
  el tiempo que tardarán en llegar al terminal de destino.

  Dada la descripción de un aeropuerto, un terminal de inicio $S$ y uno de
  destino $T$, a Lucas le gustaría saber el camino más corto para trasladarse
  desde $S$ a $T$.
  Lucas ya está cansado de perder vuelos.
  ?`Podrías ayudarlo?
\end{problemDescription}

\begin{inputDescription}
  La primera línea de la entrada contiene cuatro enteros $N$, $M$, $S$ y $T$
  ($0 < N\leq 1000, 0 < M \leq 10000$, $1 \leq S, T\leq N$).
  Los enteros $N$ y $M$ corresponden respectivamente a la cantidad de terminales
  y de caminos en el aeropuerto.
  Cada terminal es identificado con un número entre 1 y $N$.
  Los enteros $S$ y $T$ corresponden respectivamente al terminal de inicio y de
  destino.

  Posteriormente cada una de las siguientes $M$ líneas contienen tres enteros
  $a_i$, $b_i$ y $c_i$ indicando que existe un camino entre el terminal $a_i$ y
  el terminal $b_i$ y que Lucas tarda $c_i$ minutos en recorrerlo.
  Se garantiza que siempre será posible moverse entre cualquier par de
  terminales usando solamente estos caminos.

  A continuación sigue una línea con un entero $P$ ($0$) correspondiente a la
  cantidad de entradas que contiene la tabla de viajes.
  Cada una de las siguientes $P$ líneas contiene cuatro enteros $u_i$, $v_i$,
  $t_i$ y $w_i$ describiendo un viaje en tren.
  Los enteros indican que el tren viajará desde el terminal $u_i$ hasta el
  terminal $v_i$, que partirá $t_i$ minutos después de que Lucas comienza su
  recorrido y que este se demorará $w_i$ minutos en llegar al destino.
\end{inputDescription}

\begin{outputDescription}
  La salida debe contener un solo entero correspondiente al tiempo mínimo en que
  Lucas puede llegar desde $S$ a $T$.
\end{outputDescription}

\begin{scoreDescription}
  \score{10} $P=0$ y el grafo es una línea.
  \score{10} El grafo es una línea y los trenes siempre viajan entre terminales
  consecutivos.
  \score{10} $P$ = 0
  \score{10} Sin restricciones adicionales.
\end{scoreDescription}

\begin{sampleDescription}
\sampleIO{sample-1}
\sampleIO{sample-2}
\end{sampleDescription}

\end{document}
