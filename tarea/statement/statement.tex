\documentclass{oci}
\usepackage{array}
\usepackage[utf8]{inputenc}
\usepackage{lipsum}
\usepackage{colortbl}

\title{La tarea de matemática}

\begin{document}
\begin{problemDescription}
  Un fatídico día en que la profesora de matemática ya estaba aburrida de hacer
  clases, no encontró mejor idea que dar a sus alumnos la tarea de escribir en
  el cuaderno los números del uno al un millón.

  Después de algunas horas escribiendo, los alumnos notaron
  que la tarea era titánica. ?`Cuántas hojas serían necesarias para escribir
  todos los números hasta un millón? ?`Entraría todo en un solo cuaderno? 
  ?`Sería siquiera posible escribir en el cuaderno hasta cien mil?

  Un cuaderno clásico de cuadro grande tiene \textbf{28 cuadros a lo ancho y 35
    a lo alto}. Podemos suponer que en cada cuadro se escribe un solo dígito y
  que dos números consecutivos se separan con una coma (que ocupa tan
  poco espacio que es despreciable). Por otro lado un número debe ser escrito completo en
  una sola línea. Si no queda espacio en la línea actual se deben dejar los
  cuadros restantes en blanco y continuar en la siguiente línea (o página).
  La siguiente figura muestra cómo quedaría la primera página si se llena con la
  mayor cantidad de números posible.
  
  \begin{center}
  \newcolumntype{L}[1]{>{\raggedright\let\newline\\\arraybackslash}m{1em}|}
\scalebox{0.77}{
\def\arraystretch{0.9}
\setlength{\tabcolsep}{0.1em}
\begin{tabular}{|L|L|L|L|L|L|L|L|L|L|L|L|L|L|L|L|L|L|L|L|L|L|L|L|L|L|L|L|}
\hline
1,&2,&3,&4,&5,&6,&7,&8,&9,&1&0,&1&1,&1&2,&1&3,&1&4,&1&5,&1&6,&1&7,&1&8,&\\
\hline
1&9,&2&0,&2&1,&2&2,&2&3,&2&4,&2&5,&2&6,&2&7,&2&8,&2&9,&3&0,&3&1,&3&2\\
\hline
3&3,&3&4,&3&5,&3&6,&3&7,&3&8,&3&9,&4&0,&4&1,&4&2,&4&3,&4&4,&4&5,&4&6\\
\hline
4&7,&4&8,&4&9,&5&0,&5&1,&5&2,&5&3,&5&4,&5&5,&5&6,&5&7,&5&8,&5&9,&6&0\\
\hline
6&1,&6&2,&6&3,&6&4,&6&5,&6&6,&6&7,&6&8,&6&9,&7&0,&7&1,&7&2,&7&3,&7&4\\
\hline
7&5,&7&6,&7&7,&7&8,&7&9,&8&0,&8&1,&8&2,&8&3,&8&4,&8&5,&8&6,&8&7,&8&8\\
\hline
8&9,&9&0,&9&1,&9&2,&9&3,&9&4,&9&5,&9&6,&9&7,&9&8,&9&9,&1&0&0,&1&0&1\\
\hline
1&0&2,&1&0&3,&1&0&4,&1&0&5,&1&0&6,&1&0&7,&1&0&8,&1&0&9,&1&1&0&\\
\hline
1&1&1,&1&1&2,&1&1&3,&1&1&4,&1&1&5,&1&1&6,&1&1&7,&1&1&8,&1&1&9&\\
\hline
1&2&0,&1&2&1,&1&2&2,&1&2&3,&1&2&4,&1&2&5,&1&2&6,&1&2&7,&1&2&8&\\
\hline
1&2&9,&1&3&0,&1&3&1,&1&3&2,&1&3&3,&1&3&4,&1&3&5,&1&3&6,&1&3&7&\\
\hline
1&3&8,&1&3&9,&1&4&0,&1&4&1,&1&4&2,&1&4&3,&1&4&4,&1&4&5,&1&4&6&\\
\hline
1&4&7,&1&4&8,&1&4&9,&1&5&0,&1&5&1,&1&5&2,&1&5&3,&1&5&4,&1&5&5&\\
\hline
1&5&6,&1&5&7,&1&5&8,&1&5&9,&1&6&0,&1&6&1,&1&6&2,&1&6&3,&1&6&4&\\
\hline
1&6&5,&1&6&6,&1&6&7,&1&6&8,&1&6&9,&1&7&0,&1&7&1,&1&7&2,&1&7&3&\\
\hline
1&7&4,&1&7&5,&1&7&6,&1&7&7,&1&7&8,&1&7&9,&1&8&0,&1&8&1,&1&8&2&\\
\hline
1&8&3,&1&8&4,&1&8&5,&1&8&6,&1&8&7,&1&8&8,&1&8&9,&1&9&0,&1&9&1&\\
\hline
1&9&2,&1&9&3,&1&9&4,&1&9&5,&1&9&6,&1&9&7,&1&9&8,&1&9&9,&2&0&0&\\
\hline
2&0&1,&2&0&2,&2&0&3,&2&0&4,&2&0&5,&2&0&6,&2&0&7,&2&0&8,&2&0&9&\\
\hline
2&1&0,&2&1&1,&2&1&2,&2&1&3,&2&1&4,&2&1&5,&2&1&6,&2&1&7,&2&1&8&\\
\hline
2&1&9,&2&2&0,&2&2&1,&2&2&2,&2&2&3,&2&2&4,&2&2&5,&2&2&6,&2&2&7&\\
\hline
2&2&8,&2&2&9,&2&3&0,&2&3&1,&2&3&2,&2&3&3,&2&3&4,&2&3&5,&2&3&6&\\
\hline
2&3&7,&2&3&8,&2&3&9,&2&4&0,&2&4&1,&2&4&2,&2&4&3,&2&4&4,&2&4&5&\\
\hline
2&4&6,&2&4&7,&2&4&8,&2&4&9,&2&5&0,&2&5&1,&2&5&2,&2&5&3,&2&5&4&\\
\hline
2&5&5,&2&5&6,&2&5&7,&2&5&8,&2&5&9,&2&6&0,&2&6&1,&2&6&2,&2&6&3&\\
\hline
2&6&4,&2&6&5,&2&6&6,&2&6&7,&2&6&8,&2&6&9,&2&7&0,&2&7&1,&2&7&2&\\
\hline
2&7&3,&2&7&4,&2&7&5,&2&7&6,&2&7&7,&2&7&8,&2&7&9,&2&8&0,&2&8&1&\\
\hline
2&8&2,&2&8&3,&2&8&4,&2&8&5,&2&8&6,&2&8&7,&2&8&8,&2&8&9,&2&9&0&\\
\hline
2&9&1,&2&9&2,&2&9&3,&2&9&4,&2&9&5,&2&9&6,&2&9&7,&2&9&8,&2&9&9&\\
\hline
3&0&0,&3&0&1,&3&0&2,&3&0&3,&3&0&4,&3&0&5,&3&0&6,&3&0&7,&3&0&8&\\
\hline
3&0&9,&3&1&0,&3&1&1,&3&1&2,&3&1&3,&3&1&4,&3&1&5,&3&1&6,&3&1&7&\\
\hline
3&1&8,&3&1&9,&3&2&0,&3&2&1,&3&2&2,&3&2&3,&3&2&4,&3&2&5,&3&2&6&\\
\hline
3&2&7,&3&2&8,&3&2&9,&3&3&0,&3&3&1,&3&3&2,&3&3&3,&3&3&4,&3&3&5&\\
\hline
3&3&6,&3&3&7,&3&3&8,&3&3&9,&3&4&0,&3&4&1,&3&4&2,&3&4&3,&3&4&4&\\
\hline
3&4&5,&3&4&6,&3&4&7,&3&4&8,&3&4&9,&3&5&0,&3&5&1,&3&5&2,&3&5&3&\\
\hline
\end{tabular}
}
  \end{center}
  
  Dado un entero $N$, tu tarea es encontrar la cantidad de páginas necesarias para
  escribir todos los números desde el 1 hasta el $N$.
\end{problemDescription}

\begin{inputDescription}
  La entrada corresponde a un único entero $N$, tal que $N$ es mayor o igual a $1$ y menor o igual a $1.000.000$. 
\end{inputDescription}

\begin{outputDescription}
  La salida debe contener una única línea con un entero correspondiente a la
  cantidad de páginas necesarias para escribir todos los números del 1 al $N$ siguiendo las reglas
  descritas en el enunciado.
\end{outputDescription}

\section*{Subtareas y puntaje}
En este problema no hay subtareas. Se dará 4 puntos por cada caso de prueba
correcto.

\begin{sampleDescription}
  \sampleIO{sample-1} \sampleIO{sample-2} \sampleIO{sample-3}
\end{sampleDescription}

\end{document}
